\documentclass[12pt]{article}
\usepackage{titling}
\usepackage[margin=1in]{geometry}
\usepackage{titlesec}
\usepackage{blindtext}
\usepackage{xcolor}
\usepackage{makeidx}
\usepackage{hyperref}
\usepackage{tocloft}
\usepackage{siunitx}
\usepackage{listings}
\usepackage{tikz}
\usetikzlibrary{shapes.geometric}
\usetikzlibrary{arrows, decorations.markings}
\usepackage{ulem}

%%% per math functions 
\sisetup{output-exponent-marker=\ensuremath{\mathrm{e}}}

%%%%%%%%%%%%%%%%%%%%%%%%%%%%%%%%
%%%%        CODE BOX        %%%%
%%%%%%%%%%%%%%%%%%%%%%%%%%%%%%%%
\definecolor{identifier}{RGB}{250,250,250}
\definecolor{keyWords}{RGB}{207, 128, 19}
\definecolor{comments}{RGB}{37, 112, 4}
\definecolor{strings}{RGB}{59, 86, 219}


%%%%%%%%%%%%%%%%%%%%%%%%%%%%%%%%%%%%%
%%%%        DEFINE STYLES        %%%%
%%%%%%%%%%%%%%%%%%%%%%%%%%%%%%%%%%%%%
\definecolor{font}{RGB}{220,220,220}
\definecolor{grayBG}{RGB}{75,75,75}
\definecolor{codeBG}{RGB}{25,25,25}
\definecolor{azure}{RGB}{24, 147, 211}
\hypersetup{colorlinks,breaklinks, linkcolor=font}

%%%%%%%%%%%%%%%%%%%%%%%%%%%%%%%%%%%%
%%%%        NEW COMMANDS        %%%%
%%%%%%%%%%%%%%%%%%%%%%%%%%%%%%%%%%%%
\newcommand{\code}[1]{\colorbox{codeBG}{\texttt{#1}}}
%% Index title
\newcommand{\IndexTitle}[1]{
\begin{center}
    {\Huge\bfseries{#1}}
\end{center}}

%%%%%%%%%%%%%%%%%%%%%%%%%%%%%%%%%%%%%%
%%%%        RENEW COMMANDS        %%%%
%%%%%%%%%%%%%%%%%%%%%%%%%%%%%%%%%%%%%%
%% rinomino il Contents del toc in Index
\renewcommand\contentsname{\IndexTitle{Editing Utilities}}
%% aggiungo i punti dopo gli indici
\renewcommand{\cftsecleader}{\cftdotfill{\cftdotsep}}
% aumento le dimensioni del font del toc
\renewcommand\cftsecfont{\Large}
% font
\renewcommand{\familydefault}{\sfdefault}
% pagenumber color
\renewcommand{\thepage}{\textcolor{font}{\arabic{page}}}

%%%%%%%%%%%%%%%%%%%%%%%%%%%%%%%%%%
%%%%        INIZIO DOC        %%%%
%%%%%%%%%%%%%%%%%%%%%%%%%%%%%%%%%%

\makeindex
\begin{document}
\pagecolor{grayBG}
\color{font}
\tableofcontents
\addtocontents{toc}{\protect\contentsline{section}{Index}{\thepage}{Index}}
\clearpage{}
\titleformat{\section}
{\huge\bfseries}
{\thesection.}
{0.3em}
{\filcenter}[\titlerule\vspace{15px}]


\subsubsection{}

%%%%%%%%%%%%%%%%%%%%%%%%%%%%%%%%%%%%%%%%%%
\section{Kdenlive}

\subsection{Info \& Setup}

\begin{center}
    Version:   22.08.3b-x86\_64  -- Extracted AppImage
\end{center}


Appimage custom paths:
\begin{itemize}
    \item Temporary files: /home/federico/Videos/Edit/Kden/kdenFiles/temp
    \item Capture folder: /home/federico/Videos/Edit/Kden/kdenFiles/captures
    \item Library: /home/federico/Videos/Edit/Kden/kdenFiles/library
    \item Rendering, title and scripts: /home/federico/Videos/Edit/VideoRendering
    \item Effects: /home/federico/Videos/Edit/Kden/kdenFiles/data/kdenlive/effects
    \item Proxy: /home/federico/Videos/Edit/Kden/render
    \item Favorites \& other profile settings: /home/federico/Videos/Edit/Kden/kdenFiles/config/kdenlive-appimagerc
\end{itemize} 



%%%%%%%%%%%%%%%%%%%%%%%%%%%%%%%%%%%%%%%%%%%
\clearpage{}
\subsubsection{Profiles}

Default settings:
\begin{itemize}
    \item Project defaults MKV: BEST Timeline Preview
    \begin{lstlisting}
    f=matroska
    r=30
    s=1920x1080
    vcodec=libx264
    preset=ultrafast
    tune=fastdecode
    intra=1
    qscale=1
    pix_fmt=yuv420p
    threads=0
    \end{lstlisting}

    \item Proxy default MKV: BEST H.264 Proxy

    \begin{lstlisting}
    -vf "scale=640:-2" \
    -c:v libx264 \
    -preset veryfast \
    -tune fastdecode \
    -crf 15 \
    -g 30 \
    -bf 0 \
    -pix_fmt yuv420p \
    -threads 0 \
    -c:a aac \
    -b:a 128k
    \end{lstlisting}

\end{itemize}







\clearpage{}
%%%%%%%%%%%%%%%%%%%%%%%%%%%%%%%%%%%%%%%%%%

\subsubsection{Best practice}


\begin{itemize}
    \item NON USARE MAI I TITOLI -- possono buggarsi e/o far buggare il progetto
    \item Rimanere sulla 22.04 / .08 -- le versioni dopo sono instabili
    \item Attivare i proxy ma lasciarli manuali -- mettendoli automatici crea i proxy anche delle clip da 3 secondi, crashando ogni volta e creando tanta cache
    \item Per provare a fixare i crash cancellare la cartella dei config 
    \item In caso di corruzione dei proxy, si possono ricreare cancellando i file rotti (solitamente i file piccoli sono rotti/corrotti)
    \item In caso di corruzione/bug delle preview si possono cancellare (manage cached data - current project). In questo modo riavviando il progetto, verrano ri-create da 0
\end{itemize}



CURRENTLY NOT WORKING (using custom profiles)
To add preview render points:
\begin{itemize}
    \item ``I'' to add start point
    \item ``O'' to add end point  
    \item Keep it on `Preview using proxy clips'
    \item Turn on `Automatic preview' (automatic update when editing)
\end{itemize}









\clearpage{}
%%%%%%%%%%%%%%%%%%%%%%%%%%%%%%%%%%%%%%%%%%
\subsection{Effects}

\vspace{10px}

\subsubsection{Effects rules \& Kden behavior}
\begin{center}
    Per creare MACRO con più effetti contemporaneamente, {\color{red}NON} si possono usare effetti personalizzati per creare lo STACK.    
\end{center}
Funzioneranno, ma si buggeranno al prossimo riavvio di kden, che non vedrà più l'effetto e non sarà possible utilizzarlo (rimane come xml nella cartella e non causa problemi al progetto).



\vspace{20px}



\subsubsection{Video}

Main:
\begin{itemize}
    \item Transfrom - zoom, opcaity, rotation
    \item Zoompan - zoom
    \item Chroma key Advanced - per togliere green screen
    \item Rotoscopic Mask -  ritaglio a mano libera, modifiche frame 
\end{itemize}

\vspace{15px}

Other:
\begin{itemize}
    \item Vignette Effect - ombra ai bordi dello schermo
    \item Brightness - luminosità
    \item Saturation - saturazione (per black \& white)
    \item Elastic scale filter - stretch del frame 
    \item Crop scale and tilt - square crop
    \item nosync0r - transazione verticale pulita (stile cassetta)
    \item Alpha strobing - lampeggi del frame
    \item Color effect - filtri basici
    \item Letterbox - barre orizzontali
    \item Rotate keyframable - ruota frame
    \item Vertigo - movimento di frame casuale con zoom 
    \item Distort - effetti stile `specchi strani' (combo con Vertigo per flashback)
    \item Pixelize - rende il frame pixellato (keyframable)
\end{itemize}




\vspace{20px}
\subsubsection{Audio}


\begin{itemize}
    \item Band Pass - ovatta audio -- filter-width 5000.0
    \item Audio Pan - direzionamento audio (panning)
    \item Sox Gain  - aumento gain (stile earrape)
    \item Rubberband octave shift - modifiche ottave (vocina / vocione)
\end{itemize}




\clearpage{}
%%%%%%%%%%%%%%%%%%%%%%%%%%%%%%%%%%%%%%%%%%

\section{GIMP}

Impostare `Move layer', `Aligment' e `Crop' su ``Active Layer Only''.\\
\\Se il layer non supporta l'empty background: Add Alpha Channel to layer.

\vspace{20px}

Pencil settings:
\begin{itemize}
    \item Smooth stroke:
    \begin{itemize}
        \item Quality -- 100
        \item Weight -- 1000,0
    \end{itemize}
\end{itemize}



\vspace{20px}

Text border steps:
\begin{itemize}
    \item Standard:
\begin{enumerate}
    \item Right click on text layer
    \item Alpha to selection
    \item Select -- Grow
    \item New Layer -- modify the layer
    \item Swap order with layers (if they are mixed)
    \item Right click on canvas -- Select -- None
\end{enumerate}

    \item Custom:
    \begin{enumerate}
        \item Right click on text layer -- Duplicate Layer
        \item Modify the font color / change the brightness to 0 and/or black to max
        \item Move the font to create a shadow
    \end{enumerate}

\end{itemize}






\clearpage{}
%%%%%%%%%%%%%%%%%%%%%%%%%%%%%%%%%%%%%%%%%%
\section{Audacity}

Usato soprattutto per il riverbero

\vspace{20px}

Steps:
\begin{enumerate}
    \item Select the empty space after the audio clip
    \item Generate -- Silence
    \item Select the end of the first clip and the beginning of the other
    \item Right click -- Join the clips
    \item Select the audio from the clip for which you want the reverb
    \item Effect-- Delay and reverb -- Reverb
\end{enumerate}










\clearpage{}
%%%%%%%%%%%%%%%%%%%%%%%%%%%%%%%%%%%%%%%%%%
\section{Other utilities}

\vspace{20px}

\subsection{App}

Parabolic - youtube converter \& downloader 

Warpinator - File transfer


\subsection{Sites}

Remove.bg - background remover - https://www.remove.bg/upload

DS Macro - souls captions (death, area\dots) - https://rezuaq.be/new-area/image-creator/

noTube - youtube converter

convertio - file converter - https://convertio.co/it/

\end{document}
